\documentclass{article}
\usepackage{amsmath,amssymb}

\begin{document}

\textbf{Question.}  
Give an example of an open set in $\mathbb{R}^2$ (with the product metric) that is not of the form $U \times V$, where $U,V \subset \mathbb{R}$ are open.

\medskip

\textbf{Example.}  
Let
\[
P = (\mathbb{R}, |\cdot|), \qquad Q = (\mathbb{R}, |\cdot|)
\]
.  
Consider the product space
\[
P \times Q = (\mathbb{R}^2, d),
\]
where $d$ is the max (product) metric defined by
\[
d\big((x_1,y_1),(x_2,y_2)\big) = \max\{|x_1-x_2|,\; |y_1-y_2|\}.
\]

Let
\[
A = \overline{B}_{\|\cdot\|_2}\big((0,0),1\big)
   = \{(x,y) \in \mathbb{R}^2 : x^2 + y^2 \le 1\},
\]
the closed unit ball in $\mathbb{R}^2$ with respect to the Euclidean norm.

\medskip

\textbf{Claim 1.}  
$A^c$ is open in $(\mathbb{R}^2,d)$.

\medskip

\textbf{Proof.}  
Let $(a,b) \in A^c$. Since $A^c$ is open in $(\mathbb{R}^2,\|\cdot\|_2)$, there exists $r>0$ such that
\[
B_{\|\cdot\|_2}\big((a,b), r\big) \subseteq A^c.
\]

If $(x,y) \in B_{\|\cdot\|_2}\big((a,b),r\big)$, then
\[
(x,y) \in A^c, \quad \text{so} \quad \sqrt{x^2 + y^2} > 1,
\]
and
\begin{equation}
\sqrt{(x-a)^2 + (y-b)^2} < r.
\tag{$\ast$}
\end{equation}
Hence,
\[
(x-a)^2 + (y-b)^2 < r^2.
\]

Therefore, if we take
\[
(x-a)^2 < \frac{r^2}{2}
\quad \text{and} \quad
(y-b)^2 < \frac{r^2}{2},
\]
then $(\ast)$ is satisfied and
\[
(x,y) \in B_{\|\cdot\|_2}\big((a,b),r\big) \subseteq A^c.
\]

This forces,
\[
|x-a| < \frac{r}{\sqrt{2}}
\quad \text{and} \quad
|y-b| < \frac{r}{\sqrt{2}}.
\]

Let $r' = \frac{r}{\sqrt{2}}$. Then
\[
\max\{|x-a|, |y-b|\} < r',
\]
and hence
\[
d\big((a,b),(x,y)\big) < r'.
\]

Therefore,
\[
B_d\big((a,b), r'\big) \subseteq A^c.
\]
Thus, $A^c$ is open in $(\mathbb{R}^2, d)$.

\hfill$\square$

\medskip

\textbf{Claim 2.}  
$A^c$ cannot be written in the form $U \times V$, where $U \subset \mathbb{R}$ and $V \subset \mathbb{R}$ are open.

\medskip

\textbf{Proof.}  
Assume, for contradiction, that
\[
A^c = U \times V
\]
for some open sets $U,V \subset \mathbb{R}$.

Consider the horizontal line $\{(x,2) : x \in \mathbb{R}\}$.  
Since $x^2 + 2^2 > 1$ for all $x \in \mathbb{R}$, this entire line lies in $A^c$.  
Thus, for every $x \in \mathbb{R}$,
\[
(x,2) \in U \times V,
\]
which implies $\mathbb{R} \subset U$.

Similarly, consider the vertical line $\{(2,y) : y \in \mathbb{R}\}$.  
Again, $2^2 + y^2 > 1$ for all $y \in \mathbb{R}$, so this entire line lies in $A^c$, implying
\[
\mathbb{R} \subset V.
\]

Hence,
\[
U = \mathbb{R}, \qquad V = \mathbb{R},
\]
and therefore
\[
U \times V = \mathbb{R}^2,
\]
which contradicts $A^c \neq \mathbb{R}^2$.

Thus $A^c$ cannot be written as a product of open sets.
\hfill$\square$

\end{document}
