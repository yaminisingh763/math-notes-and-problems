\documentclass[12pt]{article}

\usepackage{amsmath, amssymb}
\usepackage{amsthm}
\usepackage{graphicx}
\usepackage{setspace}

\setstretch{1.15}
\setlength{\parindent}{0pt}
\setlength{\parskip}{0.6em}

\title{\textbf{FERMAT'S little THEOREM}}
\author{Yamini Singh}
\date{January 2026}

\begin{document}

\maketitle

\hrule
\vspace{1em}

\noindent
\textbf{Fermat's Little Theorem.}
Let $p$ be a prime number and let $a$ be any integer. Then
\[
a^p \equiv a \pmod{p},
\]
that is,
\[
p \mid (a^p - a).
\]

\vspace{3em}

\begin{proof}

\vspace{1em}

\textbf{Base step:} $a = 1$

\[
1^p - 1 = 0,
\]
hence $p \mid (1^p - 1)$.

\vspace{1em}

\textbf{Induction hypothesis:}

Assume that
\[
p \mid \big( (a-1)^p - (a-1) \big).
\]

\vspace{1em}

\textbf{Induction step:}

Let us suppose we have a number of colours.

Consider a set of necklaces with $p$ beads, each of which can be coloured with any of the $a$ colours.

\vspace{0.5em}

We partition this set using the following strategy.

Choose any colour, say $x$, from the $a$ choices. The number of beads of colour $x$ satisfies
\[
0 \le \text{number of beads of colour } x \le p.
\]

Let $A_i$ be the set of necklaces having exactly $i$ beads of colour $x$.

Then,
\[
\bigcup_{i=0}^{p} A_i = \text{the set of all necklaces possible}.
\]

The cardinality of this union is
\[
\left| \bigcup_{i=0}^{p} A_i \right| = a^p.
\]

\vspace{0.5em}

For $1 \le i \le p-1$, the cardinality of $A_i$ is
\[
|A_i| = \binom{p}{i} (a-1)^{p-i}.
\]

For the case $i = p$, there is one monochromatic necklace of colour $x$.

For $i = 0$, we have
\[
|A_0| = (a-1)^p.
\]

(Note that $A_0$ can still contain $(a-1)$ monochromatic necklaces.)

\vspace{0.5em}

Observe that

\[
\sum_{i=1}^{p-1} \binom{p}{i} (a-1)^{p-i}
+ (a-1)^p + 1 = a^p.
\]

Removing the monochromatic necklaces, which appear only in $A_0$ and $A_p$, we obtain

\[
\sum_{i=1}^{p-1} \binom{p}{i} (a-1)^{p-i}
+ \big( (a-1)^p - (a-1) \big).
\]

Hence,
\[
\sum_{i=1}^{p-1} \binom{p}{i} (a-1)^{p-i}
+ \big( (a-1)^p - (a-1) \big)
= a^p - a.
\]

\vspace{0.5em}

For $1 \le i \le p-1$, we know that $\binom{p}{i}$ is divisible by $p$.

For the last term, by the induction hypothesis,
\[
p \mid \big( (a-1)^p - (a-1) \big).
\]

\vspace{0.6em}

Since each term in the sum
\[
\sum_{i=1}^{p-1} \binom{p}{i} (a-1)^{p-i}
\]
is divisible by $p$, and
\[
p \mid \big( (a-1)^p - (a-1) \big),
\]
it follows that
\[
p \mid (a^p - a).
\]

This completes the proof of Fermat's Little Theorem.

\end{proof}

\end{document}
